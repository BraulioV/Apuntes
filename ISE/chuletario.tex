\documentclass[10pt,spanish, landscape, twocolumn]{article}

\usepackage[spanish]{babel}
\usepackage[utf8]{inputenc}
% \usepackage{amsmath, amsthm}
\usepackage{amsfonts, amssymb, latexsym}
\usepackage{enumerate}
\usepackage[usenames, dvipsnames]{color}
\usepackage{colortbl}
\usepackage[landscape]{geometry}
\usepackage{minted}

\usepackage[bookmarks=true,
            bookmarksnumbered=false, % true means bookmarks in
                                     % left window are numbered
            bookmarksopen=false,     % true means only level 1
                                     % are displayed.
            colorlinks=true,
            linkcolor=webblue]{hyperref}
\definecolor{webgreen}{rgb}{0, 0.5, 0} % less intense green
\definecolor{webblue}{rgb}{0, 0, 0.5}  % less intense blue
\definecolor{webred}{rgb}{0.5, 0, 0}   % less intense red

\setlength{\parindent}{0pt}
\setlength{\parskip}{1ex plus 0.5ex minus 0.2ex}

%Definimos autor y título
\title{Formulario ISE}
\author{Marta Gómez}

\begin{document}

% \section{Tema 3}
% \subsection{Tiempo de ejecución}
% \begin{enumerate}[$\bullet$]
%     \item \textbf{Tiempo de ejecución}
%     \begin{displaymath}
%         T_{EJEC} = NI \cdot CPI \cdot T_{RELOJ} = \frac{NI \cdot CPI}{f_{RELOJ}}
%     \end{displaymath}

%     \item \textbf{MIPS}
%     \begin{displaymath}
%         MIPS = \frac{NI}{T_{EJEC} \cdot 10^6} = \frac{f_{RELOJ}}{CPI \cdot 10^6}
%     \end{displaymath}

%     \item \textbf{MIPS Relativos}
%     \begin{displaymath}
%         MIPS_{relativos} = \frac{T_{EJEC~MAQUINA~REF}}{T_{EJEC}} \cdot MIPS_{MAQUINA~REF}
%     \end{displaymath}

%     \item \textbf{MFLOPS}
%     \begin{displaymath}
%         MFLOPS = \frac{Operaciones~de~coma~flotante}{T_{EJEC}\cdot 10^6}
%     \end{displaymath}
% \end{enumerate}

% \newpage

\section{Tema 4}
\subsection{Conceptos básicos}
\begin{enumerate}[$\bullet$]
    \item $W$, \textbf{Tiempo de espera en cola} o \textit{waiting time}
    \item $S$, \textbf{Tiempo de servicio} o \textit{service time}
    \begin{displaymath}
        S_i = \frac{B_i}{C_i}
    \end{displaymath}
    \item $R$, \textbf{Tiempo de respuesta de la estación de servicio} o \textit{response time}
    \begin{displaymath}
        R = W + S     
    \end{displaymath}
    \item $Z$, \textbf{Tiempo de reflexión} o \textit{think time} (sólo en redes cerradas)
    \item $N$, \textbf{Número de trabajos}, incluye tanto en la cola como en el recurso
    \begin{displaymath}
        N_0 = X_0 \cdot R_0
    \end{displaymath}
    \item $N_T$, \textbf{Número de trabajos total} (sólo en redes cerradas)
    \begin{displaymath}
        N_T = N_Z + N_0 = X_0 \cdot Z + X_0 \cdot R_0 = X_0 \cdot (Z + R_0)
    \end{displaymath}
    \begin{enumerate}[$\longrightarrow$]
        \item En red cerrada tipo batch
        \begin{displaymath}
            N_T = N_0
        \end{displaymath}
        \item En red cerrada tipo interactivo
        \begin{displaymath}
            N_T = N_0 + N_Z
        \end{displaymath}
    \end{enumerate}
    \item $N_Z$, \textbf{Número de trabajos en reflexión} (sólo en redes cerradas)
    \begin{displaymath}
        N_Z = X_0 \cdot Z
    \end{displaymath}
    \item $T$, \textbf{Duración del periodo de medida}
    \item $A$, \textbf{Número de trabajos solicitados} o \textit{arrivals}
    \item $C$, \textbf{Número de trabajos completados} o \textit{completions}
    \item $B_i$, \textbf{Tiempo que un dispositivo está ocupado} o \textit{busy time}
    \item $\lambda$, \textbf{Tasa de llegada} o \textit{arrival rate}
    \begin{displaymath}
        \lambda_i = \frac{A_i}{T}
    \end{displaymath}
    \item $\tau$, \textbf{Tiempo medio entre llegadas} o \textit{interarrival time}
    \begin{displaymath}
        \tau_i = \frac{1}{\lambda_i} = \frac{T}{A_i}
    \end{displaymath}
    \item $X$, \textbf{Productividad} o \textit{throughput}
    \begin{displaymath}
        X_i = \frac{C_i}{T}
    \end{displaymath}
    \item $Q_i$, \textbf{Número medio de trabajos en cola} o \textit{jobs in queue}
    \item $U$, \textbf{Número medio de trabajos servidos} y \textbf{utilización}
    \begin{displaymath}
        U_i = N_i - Q_i = \frac{B_i}{T}
    \end{displaymath}
    \item $V$, \textbf{Razón de visita} o \textit{visit ratio}
    \begin{displaymath}
        V_i = \frac{C_i}{C_0}
    \end{displaymath}
    \item $D$, \textbf{Demanda de servicio} o \textit{service demand}
    \begin{displaymath}
        D_i = \frac{B_i}{C_0} = V_i \cdot S_i
    \end{displaymath}
\end{enumerate}

\newpage
\subsection{Leyes operacionales}
\begin{enumerate}[$\bullet$]
    \item \textbf{Hipótesis del equilibrio de flujo}
    \begin{displaymath}
        A_0 \approx C_0 \qquad\ \lambda_0 \approx X_0 \qquad\ \frac{| A_i - C_i |}{C_i} \approx 0
    \end{displaymath}

    \item \textbf{Ley de Little}
    \begin{displaymath}
        N_0 = \lambda_0 \cdot R_0 \underbrace{= X_0 \cdot R_0}_{no ~ saturado} \qquad\ \underbrace{Q_i = \lambda_i \cdot W_i = X_i \cdot W_i}_{cola ~ dispositivo}
    \end{displaymath}

    \item \textbf{Ley de la Utilización}
    \begin{displaymath}
        U_i = \frac{B_i}{T} = \frac{C_i}{T} \cdot \frac{B_i}{C_i} = X_i \cdot S_i \underbrace{= \lambda_i \cdot S_i}_{no ~ saturado}
    \end{displaymath}

    \item \textbf{Ley del flujo forzado}
    \begin{displaymath}
        V_i = \frac{C_i}{C_0} = \frac{X_i}{X_0} \qquad\ \Longrightarrow \qquad\ X_i = X_0 \cdot V_i
    \end{displaymath}
    \begin{displaymath}
        U_i = X_i \cdot S_i = X_0 \cdot V_i \cdot S_i = X_0 \cdot D_i = \lambda_0 \cdot D_i
    \end{displaymath}

    \item \textbf{Ley general del tiempo de respuesta}
    \begin{displaymath}
        R_0 = \sum_{i=1}^K V_i \cdot R_i
    \end{displaymath}

    \item \textbf{Ley del tiempo de respuesta interactivo}
    \begin{displaymath}
        R_0 = \frac{N_T}{X_0} - Z
    \end{displaymath}
\end{enumerate}

\newpage
\subsection{Límites optimistas del rendimiento}
\begin{enumerate}[$\bullet$]
    \item \textbf{Productividad máxima}
    \begin{displaymath}
        X_0^{max} = \frac{1}{D_b}
    \end{displaymath}

    \item \textbf{Tiempo de respuesta mínimo}
    \begin{displaymath}
        R_0^{min} = \sum_{i=1}^K V_i \cdot S_i = \sum_{i=1}^K D_i \equiv D \qquad\ \textrm{Porque $R_i^{min} = S_i$}
    \end{displaymath}
\end{enumerate}

\subsubsection{En redes cerradas}
\begin{enumerate}[$\bullet$]
    \item \textbf{Tiempo de respuesta}, dependendiendo de si hay mucha carga o poca
    \begin{displaymath}
        R_0 \geq max \{D, N_T \cdot D_b - Z\}
    \end{displaymath}

    \item \textbf{Productividad}, dependendiendo de si hay mucha carga o poca
    \begin{displaymath}
        X_0 \leq min \Bigg \{ \frac{N_T}{D + Z}, \frac{1}{D_b} \Bigg\}
    \end{displaymath}

    \item \textbf{Punto teórico de saturación}
    \begin{displaymath}
        D = N_T^* \cdot D_b - Z \qquad\ \Longrightarrow \qquad\ N_T^* = \frac{D+Z}{D_b}
    \end{displaymath}
\end{enumerate}

\newpage
\subsection{Resolución de redes de colas}
\begin{enumerate}[$\bullet$]
    \item \textbf{Tiempo de respuesta medio}
    \begin{displaymath}
        R_i = N_i \cdot S_i + S_i = (N_i + 1) \cdot S_i = (X_i \cdot R_i + 1) \cdot S_i = X_i \cdot S_i \cdot R_i + S_i
    \end{displaymath}
    \begin{displaymath}
        R_i - X_i \cdot S_i \cdot R_i = S_i \qquad\ \Longrightarrow \qquad\ R_i \cdot (1 - X_i \cdot S_i) = S_i
    \end{displaymath}
    \begin{displaymath}
        R_i = \frac{S_i}{1 - X_i \cdot S_i} = \frac{S_i}{1 - U_i} = \frac{S_i}{1 - X_0 \cdot D_i}
    \end{displaymath}

    \item \textbf{Modelos abiertos}
    \begin{enumerate}[1.]
        \item Calcular demanda de servicio de cada estación
        \begin{displaymath}
            D_i = V_i \cdot S_i
        \end{displaymath}
        \item Calcular tiempo de respuesta de cada estación
        \begin{displaymath}
            R_i = \frac{S_i}{1 - X_0 \cdot D_i} = \frac{S_i}{1 - \lambda_0 \cdot D_i}
        \end{displaymath}
        \item Calcular tiempo de respuesta del servidor
        \begin{displaymath}
            R_0 = \sum_{i=1}^K V_i \cdot R_i = \sum_{i=1}^K \frac{V_i \cdot S_i}{1 - \lambda_0 \cdot D_i} = \sum_{i=1}^K \frac{D_i}{1 - \lambda_0 \cdot D_i}
        \end{displaymath}
        \item Calcular resto de valores de forma normal
    \end{enumerate}
\end{enumerate}

\newpage
\subsection{\texttt{solvenet}}
\begin{minted}{text}
Usage: solvenet [0|1] [lambda0 | NT Z] K S1 V1...SK VK
    With no parameters, shows this message
    network: 0 (open) and 1 (closed)
    lambda0: arrival rate = throughput (only open networks)
    NT:      total number of jobs in the net (only closed networks)
    Z:       think time (only interactive closed networks)
    K:       number of service stations
    Si:      service time of device i
    Vi:      ratio visit of device i    
\end{minted}

\end{document}